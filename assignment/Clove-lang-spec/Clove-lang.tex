\documentclass[12pt,a4paper]{article}
\usepackage[
  a4paper, % Set the paper size -- the document class only affects plainTeX.
  top=1.0cm,
  right=2.0cm,
  bottom=1.0cm,
  left=2.0cm
]{geometry}
\pagenumbering{gobble} % Disable page numbering

%----------------------------------
% FOOTNOTE
%   - bottom -- push to the bottom of a page
%   - hang -- flushes the footnote marker to the left margin of the page
%   - flushmargin -- flushes the text to the left margin as well
%----------------------------------
\usepackage[bottom,hang,flushmargin]{footmisc}


%----------------------------------
% LINE SPACING, POSITION SETTING
%----------------------------------
\usepackage{setspace}
\usepackage{graphicx}

%----------------------------------
% CHARACTERS, LANGUAGE, AND FORMATTING PRESETS
%----------------------------------
\usepackage[utf8]{inputenc}
\usepackage[T1]{fontenc}
\usepackage[british]{babel}

%----------------------------------
% MINTED: SOURCE CODE LISTINGS
%----------------------------------
\usepackage[draft=true,section]{minted} % TODO: Remove draft option for coloured syntax highlighting.

% REMOVE SYNTAX ERROR HIGHLIGHTING
% https://tex.stackexchange.com/a/343506
\makeatletter
\AtBeginEnvironment{minted}{\dontdofcolorbox}
\def\dontdofcolorbox{\renewcommand\fcolorbox[4][]{##4}}
\makeatother

%----------------------------------
% MISC
%----------------------------------
% PREVENT PAGE BREAK AROUND \include{}
% Usage: \include*{filename} % Remember, no extension!
% https://stackoverflow.com/a/1210233/10620237
\usepackage{newclude}

% Nested input.
% https://tex.stackexchange.com/a/60227
\usepackage{import}

%----------------------------------
% PREVENT LINE BREAK ON en-dash -- NOTE: It must be the very last package!
%   Usage: \==~
%----------------------------------
% https://tex.stackexchange.com/a/283202
\usepackage[shortcuts]{extdash}



\begin{document}
\begin{center}
  \Huge {Clove-lang}
\end{center}

\noindent Clove is a Java-based, interpreted, dynamically typed, general-purpose programming language; syntactically, it resembles JavaScript. It is primarily an imperative language, but has functional programming features, such as lambda expressions, and the programs can be composed in such ways that utilise the declarative paradigm. The language allows for defining immutable primitive values, but compound structures, such as lists, can still be changed, but not reassigned.\par

% \import{./code-snippets/}{constants.tex}

Parts of the language, such as for-loops, while-loops, if-statements, or variable and function declarations, are conventional and identical to some other languages (namely, JavaScript). Variable names cannot start with a number and must be preceded by either \emph{let} or \emph{const} tokens (both have numerous aliases).\par

Many keywords included in the language can be expressed in several ways \==~Clove-lang contains Polish translations of commonly used keywords, and can easily be expanded with other languages by slightly altering the grammar. This enables users not knowing English to start programming and familiarise themselves with the programming concepts more easily.



\subsection*{Data types}
% Clove-lang supports the following data types:

% \begin{itemize}
%   \itemsep-0.25em
%   \item integers,
%   \item rational numbers,
%   \item strings,
%   \item booleans,
%   \item anonymous objects,
%   \item lists, and
%   \item anonymous functions.
% \end{itemize}

The following data types are supported: integers, rational numbers, strings, booleans, anonymous objects, lists, and anonymous functions. The integer and rational types are in Java's \emph{long} and \emph{double} range respectively. The list type is similar in functionality to Python's \emph{List} and JavaScript's \emph{Array} \==~it is implemented using \emph{ArrayList} in Java. Anonymous function is unique in that it acts as a container for a function definition as opposed to being a standalone implementation of a type.\par

All the types evaluate to values that, in most cases, are treated the same way \==~they can be passed around, nested (if the type allows) and returned from functions. It applies to both literals and assigned/dereferenced values.



\subsection*{Declarations and definitions}
Declarations and definitions in Clove are similar to those in JavaScript and use the same keywords. To define an empty list or anonymous object, their respective empty literals must be used. Functions can be defined in two ways: with JavaScript-esque \emph{function} keyword, or as an anonymous, value-function. Examples of some common declarations and definitions below\footnote{More code snippets are available in the `demonstrations' directory. For example programs, go to `examples'.}.

\import{./code-snippets/}{declarations-and-definitions.tex}



\subsection*{Prototype functions}
The string, list and anonymous object value-types have built-in prototype functions that can be accessed using the arrow-notation.

\import{./code-snippets/}{prototype-functions.tex}



\subsection*{Data structures}
The supported data types allow for creating compound data structures. The code below demonstrates an example usage of anonymous objects and functions to imitate a class-like structure.

\import{./code-snippets/}{data-structures.tex}









\section*{Features}
Clove-lang has some useful built-in features, a couple of which allow for interaction with the outside world. This section introduces them in detail.

% \subsection*{random()}

\subsection*{file()}


\subsection*{http()}


\end{document}
