\documentclass[12pt,a4paper]{article}
\usepackage[
  a4paper, % Set the paper size -- the document class only affects plainTeX.
  top=1.0cm,
  right=1.5cm,
  bottom=1.0cm,
  left=1.5cm
]{geometry}
\pagenumbering{gobble} % Disable page numbering

%----------------------------------
% FOOTNOTE
%   - bottom -- push to the bottom of a page
%   - hang -- flushes the footnote marker to the left margin of the page
%   - flushmargin -- flushes the text to the left margin as well
%----------------------------------
\usepackage[bottom,hang,flushmargin]{footmisc}


%----------------------------------
% LINE SPACING, POSITION SETTING
%----------------------------------
\usepackage{setspace}
\usepackage{graphicx}

%----------------------------------
% CHARACTERS, LANGUAGE, AND FORMATTING PRESETS
%----------------------------------
\usepackage[utf8]{inputenc}
\usepackage[T1]{fontenc}
\usepackage[british]{babel}

%----------------------------------
% MINTED: SOURCE CODE LISTINGS
%----------------------------------
\usepackage[draft=true,section]{minted} % TODO: Remove draft option for coloured syntax highlighting.

% REMOVE SYNTAX ERROR HIGHLIGHTING
% https://tex.stackexchange.com/a/343506
\makeatletter
\AtBeginEnvironment{minted}{\dontdofcolorbox}
\def\dontdofcolorbox{\renewcommand\fcolorbox[4][]{##4}}
\makeatother

%----------------------------------
% MISC
%----------------------------------
% PREVENT PAGE BREAK AROUND \include{}
% Usage: \include*{filename} % Remember, no extension!
% https://stackoverflow.com/a/1210233/10620237
\usepackage{newclude}

% Nested input.
% https://tex.stackexchange.com/a/60227
\usepackage{import}

%----------------------------------
% PREVENT LINE BREAK ON en-dash -- NOTE: It must be the very last package!
%   Usage: \==~
%----------------------------------
% https://tex.stackexchange.com/a/283202
\usepackage[shortcuts]{extdash}



\begin{document}
\begin{center}
  \Huge {Clove-lang}
\end{center}

\noindent Clove is a Java-based, interpreted, dynamically typed, general-purpose programming language. It is primarily a procedural language, but has functional programming features, such as lambda expressions, and the programs can be composed in ways that utilise the declarative paradigm. The language allows for defining immutable primitive values, but compound structures, such as lists, can still be changed, but not reassigned; otherwise, all values are mutable.\par

Common statements and expressions, such as loops, if-statements, variable and function declarations/calls/invocations are conventional and similar to other languages, namely JavaScript. Variable names cannot start with a number and must be preceded by either \emph{let} or \emph{const} tokens.\par

Many keywords included in the language can be expressed in a bunch of ways \==~Clove-lang contains Polish translations of commonly used keywords. The translations can easily be expanded by slightly altering the grammar. These translations enable people not knowing English to start programming and familiarise themselves with programming concepts more easily.



\vspace{-0.25em}
\subsection*{Data types}
The following data types are supported: integers, rational numbers, strings, booleans, anonymous objects, lists, arrays, and anonymous functions. The integer and rational types are in Java's \emph{long} and \emph{double} ranges respectively. The list type is similar in functionality to Python's \emph{List} and JavaScript's \emph{Array}. Array includes all the list's features, but has a fixed size. Anonymous function is unique in that it is a container for a function definition instead of being a standalone implementation of a type.\par

All the types evaluate to values that are treated similarly \==~they can be passed around, nested, evaluated, and returned from functions. It applies to both literals and dereferenced values. Some operations, such as addition, can be performed on mixed types, but there are reasonable restrictions, e.g. it is impossible to multiply a string by a number, but it is fine to add a number to a string.



\vspace{-0.25em}
\subsection*{Declarations and definitions}
Declarations and definitions in Clove are similar to those in JavaScript. To define an empty list or anonymous object, their empty literals must be used. Functions can be defined in two ways: standalone, with the \emph{function} keyword, or as an anonymous \emph{value-function}. Examples of some common declarations and definitions can be seen below\footnote{More code snippets and example programs are available in the `demonstrations' and `examples' directories.}. The variables defined at the top are accessible from every subsequent block, therefore they have a `global' scope in this example. Variables in function invocations or any other block are defined in their corresponding scope (are inaccessible from the outside) and removed after the block is finished.

\import{./code-snippets/}{declarations-and-definitions.tex}



\subsection*{Prototype functions and shorthand operators}
The string, list, array and anonymous object value-types have built-in prototype functions that can be accessed using the arrow-notation. Prototype functions are invoked just like normal functions, with the parentheses after the function's name. Clove also supports a few shorthand operators for reassignment and incrementation/decrementation. Examples shown below.

\import{./code-snippets/}{prototype-functions-and-shorthand-operators.tex}



% Take some space above the heading after a code-block.
\vspace{-1.5em}
\subsection*{Data structures}
The supported data types allow for creating compound data structures. The code below demonstrates an example usage of anonymous objects and functions to imitate a class-like structure.

\import{./code-snippets/}{data-structures.tex}



\vspace{-1.5em}
\subsection*{Built-in functions}
Clove-lang has some useful built-in features, some of which interact with the outside world.

\paragraph*{get\_args()}
Returns a list of command-line arguments passed into the program.

\vspace{-0.75em}
\paragraph*{random()}
Returns a random integer or a rational number in a given range. \mintinline{JavaScript}{random(0, 3.14)} would return a random rational number in the 0 <= x < 3.14 range \==~it recognises a floating-point number in the second argument, therefore the result will also be a float.

\vspace{-0.75em}
\paragraph*{file()}
It takes three arguments: writing option, path, content. The function's behaviour depends on the first argument which specifies whether the new content should be appended to the file, or replace the old one. If the path does not exist, it will be created, although it is limited to one subdirectory.

\vspace{-0.75em}
\paragraph*{http()}
This function supports four main HTTP requests: GET, POST, PUT, and DELETE, one of which has to be specified in the first argument. The second argument is the resource's URL, and the third, optional argument is the request body (only applicable to POST and PUT requests) which is a string of url-encoded pairs, or an anonymous object.

\import{./code-snippets/}{http-request-and-file-writing.tex}



\end{document}
